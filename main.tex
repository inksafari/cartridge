% Cartridge | (c) 2016 林子楀( Lin Tzu-Yu )| CC BY-NC-SA 4.0 License
% https://github.com/inksafari/cartridge

% ****************************************************************************
% Document Class Definition( Global Settings )
% ****************************************************************************
\documentclass[%
    paper=A4,               % common are A4, B5 or custom(18.5cm:23cm)
    pagesize,               % everything gets calculated correctly
    twoside=true,           % oneside or twoside printing
    open=right,             % doublepage cleaning ends up right side
    numbers=noenddot,       % no dot/period after the last chapter number
    draft=false             % if true: no images, layout errors shown
]{scrbook}                  % “book”-like class of the KOMA-Script bundle

% ****************************************************************************
% Load and Configure Packages
% ****************************************************************************

%% Koma-Script Settings
%% ------------------------------------------------------------
%\KOMAoptions{}
%\recalctypearea                    % or use "pagesize" option

%% Colors
%% ------------------------------------------------------------
%\usepackage{color} 
\usepackage[dvipsnames*,svgnames]{xcolor} 
\selectcolormodel{natural}          % natural, cmyk or RGB 
% !TEX root = ../main.tex

% ============================================================================
% STOCKHOLMS COLOR THEME (sthlm)
% ============================================================================
%     cf. http://varumarkesmanual.stockholm.se/farger
%         https://github.com/hendryolson/MathNoteBook
%         C:Coated=光面紙質(亮面);U:Uncoated=不光面紙質
%         usage: \color{sthlmRed}
 
% GREEN
\definecolor{sthlmGreen}{RGB}{0,134,127}          % Hex #00867f
\definecolor{sthlmLightGreen}{RGB}{213,247,244}   % Hex #d5f7f4

% BLUE
\definecolor{sthlmBlue}{RGB}{0,110,191}           % Hex #006ebf
\definecolor{sthlmLightBlue}{RGB}{214,237,252}    % Hex #d6edfc

% PURPLE
\definecolor{sthlmPurple}{RGB}{93,35,125}         % Hex #5d237d
\definecolor{sthlmLightPurple}{RGB}{241,230,252}  % Hex #f1e6fc

% RED
\definecolor{sthlmRed}{RGB}{196,0,100}            % Hex #c40064
\definecolor{sthlmLightRed}{RGB}{254,222,237}     % Hex #fedeed

% ORANGE
\definecolor{sthlmOrange}{RGB}{221,74,44}         % Hex #dd4a2c
\definecolor{sthlmLightOrange}{RGB}{255,215,210}  % Hex #ffd7d2

% GREY
\definecolor{sthlmGrey}{RGB}{245,243,238}         % Hex #f5f3ee
\definecolor{sthlmDarkGrey}{RGB}{51,51,51}        % Hex #333333

% ============================================================================
% Base16
% ============================================================================
%   https://github.com/chriskempson/base16 

% GREEN
\definecolor{base0B}{HTML}{A1B52C}                % Hex #A1B52C
\definecolor{base0C}{HTML}{86C1B9}                % Hex #86C1B9

% BLUE
\definecolor{base0D}{HTML}{7CAFC2}                % Hex #7CAFC2

% PURPLE
\definecolor{base0E}{HTML}{BA8BAF}                % Hex #BA8BAF

% RED
\definecolor{base08}{HTML}{AB4642}                % Hex #AB4642

% ORANGE
\definecolor{base09}{HTML}{DC9656}                % Hex #DC9656

% ============================================================================
% DUKE COLOR PALETTE
% ============================================================================
%   http://styleguide.duke.edu/identity/color-palette/


              % Color Definitions

%% Graphpics
%% ------------------------------------------------------------ 
\usepackage{array}
\usepackage{tikz}                   
\usetikzlibrary{shapes,arrows,decorations.pathreplacing,decorations.pathmorphing}
\usetikzlibrary{mindmap}
\usepackage[raccourcis]{fast-diagram}
%\usepackage{pgfplots}
%\pgfplotsset{compat=1.12}
\usepackage[dvips,xetex]{graphicx}  % for \includegraphics{}
\graphicspath{{img/}}
\DeclareGraphicsExtensions{.png, .PNG, .jpg, .JPG, .jpeg, .JPEG, .ps, .PS, .pdf, .PDF}

%% Tables
%% ------------------------------------------------------------ 
\usepackage{tabularx, booktabs, multirow}

%% Captions
%% ------------------------------------------------------------ 
\addtokomafont{caption}{\small}
\setkomafont{captionlabel}{\sffamily\bfseries}
\KOMAoptions{captions=tableabove}

%% Framed Boxes
%% ------------------------------------------------------------ 
\usepackage{tcolorbox}              % Creating colourful boxes 
%\usepackage{mdframed}

% http://www.cnblogs.com/SivilTaram/p/4842588.html
%\newcommand{\vline}{\color{base08}\makebox[0pt]{\textbullet}\hskip-0.5pt\vrule width 1pt\hspace{\labelsep}}

%% Source Code Syntax Highlighting
%% ------------------------------------------------------------ 
% cf. babooshka-innerjourneys.blogspot.com/2015/05/pandocmarkdownpdf.html 
%     http://tex.stackexchange.com/questions/{83204, 122256}/
\usepackage{listings, accsupp}
%\usepackage{coderemarks}           % Callouts in code blocks
\lstset{
    inputencoding    = utf8,
    captionpos       = b,           % sets the caption-position to bottom
    title            =\lstname,
    frame            = leftline,
    columns          = flexible,
    numbers          = left,
    numbersep        = 5pt,
    xleftmargin      = 2em,
    xrightmargin     = 2em,
    aboveskip        = 1em,
    backgroundcolor  =\color{white},                         
    basicstyle       =\linespread{1} \normalsize\ttfamily , 
    commentstyle     =\color{base0E}\itshape,               % Hex #BA8BAF
    keywordstyle     =\color{base0D}\bfseries,              % Hex #7CAFC2
    stringstyle      =\color{base0B},                       % Hex #A1B52C
    numberstyle      =\color{Gray}\tiny\noncopynumber,
    breaklines       = true,
    extendedchars    = false,
    showspaces       = false,
    showstringspaces = false,
    showtabs         = false,
    tabsize          = 2
}

\newcommand{\noncopynumber}[1]{
    \BeginAccSupp{method=escape,ActualText={}}
    #1
    \EndAccSupp{}
}

%\usepackage{minted}
%\setminted{linenos, frame=leftline, xleftmargin=2pc, numbersep=6pt}
%\setminted[console]{linenos=false, frame=none, fontsize=\small}

%% Footnotes
%% ------------------------------------------------------------ 
\KOMAoptions{footnotes=multiple}    % or nomultiple

% <<<<<<<<<<<<<<<<<<<<<<<<<<<<<<<<<<<<<<<<<<<<<<<<<<<<<<<<<<<<<
% Typo
% <<<<<<<<<<<<<<<<<<<<<<<<<<<<<<<<<<<<<<<<<<<<<<<<<<<<<<<<<<<<< 

%% Symbols
%% ----------------------------------------------------------- 
\usepackage{fontawesome}            % Font Awesome icons

%% Math
%% ------------------------------------------------------------ 
%\usepackage{amsfonts,amsmath,amssymb}
%\usepackage[all]{onlyamsmath}
%\usepackage{mathpazo}
%\usepackage{bm}
%\usepackage{siunitx}    
%\sisetup{per-mode = symbol}

%\usepackage{microtype}             % Typographic Tuning
%\UseMicrotypeSet[protrusion]{basicmath}  % disable protrusion for tt fonts

%% Latin
%% ------------------------------------------------------------ 
% http://www.tug.dk/FontCatalogue/
%\usepackage{lmodern}               % Latin Modern
%\usepackage[T1]{fontenc}           % font types and character verification

% Sans-Serif fonts
%   - Helvetica or Nimbus Sans  
%   - TEX Gyre Heros
%   - Biolinum ( Libertine )

% Serif fonts
%   - Latin Modern
%   - Charter

\usepackage{iftex}
\ifXeTeX
    \usepackage{fontspec}
\fi
\ifLuaTeX
    \usepackage{luatexja-fontspec}
\fi
%\setmainfont[Ligatures=TeX, Numbers={OldStyle, Proportional}]{TeX Gyre Pagella}
%\setsansfont[Ligatures=TeX, Numbers={Lining, Proportional}]{TeX Gyre Heros}
%\setmainfont[Ligatures={Common,TeX}]{Tex Gyre Termes}
%\setsansfont[Ligatures={Common,TeX}]{Droid Sans}
%\setmonofont{CMU Typewriter Text}
\defaultfontfeatures{Mapping=tex-text,Scale=MatchLowercase}
 
%% Chinese, search: fc-list -f '%{family}\n' :lang=zh | sort
%% ------------------------------------------------------------
% \setCJKmainfont[BoldFont=黑體,ItalicFont=楷書]{明體/宋體}
% \usepackage{zxjatype}                                 % 日文排版
% \usepackage[ipaex]{zxjafont}  
\ifXeTeX
    \usepackage{xltxtra,xeCJK}     
    \xeCJKsetup{BoldFont,CJKmath,CJKchecksingle,xCJKecglue}
    \XeTeXlinebreaklocale "zh"
    \XeTeXlinebreakskip = 0pt plus 1pt minus 0.1pt
    %\renewcommand{\CJKunderdotcolor}{\color{Black}}               
    \setCJKmainfont[BoldFont=Adobe Heiti Std,          
                    ItalicFont=Adobe Kaiti Std                           
                  ]{Adobe Ming Std}                       
    \setCJKmonofont{KaiGen Gothic HW TW}                  
    \setCJKsansfont{KaiGen Gothic TW}       
    \newCJKfontfamily[HeiTi]\HEI{Adobe Heiti Std}       % 黑體, usage: \HEI{內文}
    \newCJKfontfamily[KaiShu]\KAI{Adobe Kaiti Std}      % 楷書 
    \newCJKfontfamily[MingTi]\MING{Adobe Ming Std}      % 宋體 / 明朝体(日本) / 明體
    \newCJKfontfamily[FSong]\FANGSONG{Adobe Fangsong Std} % 仿宋 / 宋朝体(日本)
    \defaultCJKfontfeatures{Mapping=tex-text}           % 連字符
\fi
% search: fc-list :lang=zh | sort
\ifLuaTeX                                               
    %\usepackage[ipaex]{luatexja-preset}
    \usepackage{luatexja-preset}
    \usepackage{luatexja-ruby}
    \ltjsetruby{fontcmd=\rmfamily}                      % 日文漢字(明朝体): \mcfamily
    \setmainjfont[BoldFont={Adobe Heiti Std},
                  ItalicFont={Adobe Kaiti Std}]{Adobe Ming Std}
    \setsansjfont{Adobe Fangsong Std}
    \defaultjfontfeatures{JFM=kaiming}
    \newjfontfamily\KAI{Adobe Kaiti Std}
    \newjfontfamily\FANGSONG{Adobe Fangsong Std}
\fi
\usepackage{zhnumber}
\zhnumsetup{style={Traditional,Normal}}                 % 大寫: Financial

%% Hyphenation
%% ------------------------------------------------------------
%              babel  polyglossia
% koma-Script   o        ?
% biblatex      o        o
% csquote       o        o
\usepackage[ngerman,american]{babel}                    % 英式: british
%\usepackage{polyglossia}
%\setdefaultlanguage[variant=american]{english}         % 英式: variant=uk
%\setotherlanguage[spelling=new]{german}

%% Font Size
%% ------------------------------------------------------------ 
%\usepackage{ctexsize}                % XXX: 小四號字比 12pt 小
%\zihao{-4}                           % 小四號字(約 12.05pt);五號字(約 10.54pt )
\KOMAoptions{fontsize=12pt}           % common are 10.5pt, 11pt or 12pt

%% Paragraph Formatting(請自行調整,我不確定以下拼湊出的內容是否正確)
%% ------------------------------------------------------------ 
% https://www.sharelatex.com/learn/Paragraph_formatting
% http://blog.csdn.net/xiazdong/article/details/8892092
% https://github.com/ctex-org/ctex-kit/issues/143
% 段間距:\lineskip + \parskip
% 行間距:\lineskip = \baselineskip(baseline 間距) * \baselinestretch(伸展因子?)
% \lineskiplimit:當兩行字之間的距離小於 \lineskiplimit 時,行距自動設為 \lineskip。

%設定參考:
%  - \parskip 設置為 0pt,即段間距和行間距相等(橡皮長度?)。
%  - 不建議調整 \baselinestretch 以設定行距,由於預設的 \baselineskip 大於 font size,
%    如果 \baselinestretch 設為 1.5,得到的行距實際上大於 1.5 倍。
%  - renewcommand{\baselinestretch}{1.5} 等同 \linespread{1.5}
%                                   1.0      single spacing
%                                   1.3      one-and-a-half spacing
%                                   1.6      double spacing
%  - 可透過 \fontsize 調整 \baselineskip,使 \baselinestretch 一直是 1。

%常用套件:
%  - 漢字的彈性間距(影響字間距之調整、段落左右對齊及斷行):
%    - 以一到兩字漢字寬度為佳。
%    - LuaTEX-ja:透過 kanjiskip 調整。      
%    - XeCJK:透過 \CJKglue 調整,預設: \hskip 0pt plus 0.08\baselineskip
%      以 CTeX 預設值為例(5 号字 10.54pt,1.3 倍行距,版心宽度 345pt):
%      \hskip 0pt plus 0.08\baselineskip
%      即 0.08 * \baselineskip * ( textwidth / fontsize ) / fontsize = 4漢字寬
%         0.08 * 16.445pt     * ( 345pt     / 10.54pt)   / 10.54pt
%      \baselineskip 之計算:10.54pt * 1.3 倍行距(透過 linespread 設定)* 1.2
%      重新調整:
%      \renewcommand{\CJKglue}{} 
%  - indentfirst:調整 \parindent,讓段首縮進兩漢字寬度,即 2em。
%  - setspace:調整行距,如 \sin­glespac­ing, \one­half­s­pac­ing, and \dou­blespac­ing。
%                      或 \linestretch{}, 1 為單倍, 1.2 為 1.5 倍, 1.6 為雙倍。

% http://tex.stackexchange.com/questions/161254/
%\KOMAoptions{parskip=?}            % half / full / ...
%\setparsizes{parindent}{parskip}{parfilskip}          
\usepackage{indentfirst}            % 段首縮進
\setparsizes{0pt}{0pt plus 1pt minus 0.1pt}{0pt plus 1fill} 
\setlength{\parindent}{2em}         % 段首縮進兩漢字寬度(不縮進: 0pt )
\setlength{\emergencystretch}{3em}  % prevent overfull lines 
% https://www.bdwm.net/bbs/bbstcon.php?board=MathTools&threadid=15627976       
% lineskip / baselineskip = 20 bp / (12 bp * (6 / 5)) 
\linespread{1.39}\selectfont        % 如果不加 \selectfont?

%% <<<<<<<<<<<<<<<<<<<<<<<<<<<<<<<<<<<<<<<<<<<<<<<<<<<<<<<<<<<<
%% Typo(END)
%% <<<<<<<<<<<<<<<<<<<<<<<<<<<<<<<<<<<<<<<<<<<<<<<<<<<<<<<<<<<< 

%% Lists
%% ------------------------------------------------------------
%% cf. https://github.com/hust-latex/hustthesis  
\usepackage{enumitem}   
\setlist{noitemsep,partopsep=0pt,topsep=.8ex}
\setlist[1]{labelindent=\parindent}
\setlist[description]{font=\sffamily\bfseries} 
\setlist[enumerate,1]{label=\arabic*.,ref=\arabic*}
\setlist[enumerate,2]{label*=\arabic*,ref=\theenumi.\arabic*}
\setlist[enumerate,3]{label=\emph{\alph*}),ref=\theenumii\emph{\alph*}}
% 1. 第一層
% 2. 第一層
%    2.1 第二層
%    2.2 第二層
%        a) 第三層
%        b) 第三層
%           A. 第四層
%           B. 第四層

%% Bibliography & References
%% ------------------------------------------------------------ 
\usepackage[%
    backend = biber,                
    style   = grain,               % github.com/inksafari/grain
    sorting = centy]{biblatex}     % ↘源於 biblatex-caspervector 
\addbibresource{files/books.bib}
\addbibresource{files/online.bib}

%% Quotation
%% ------------------------------------------------------------
\usepackage[
  autostyle=true,
  german=quotes,
  english=american]{csquotes}

%% Index 
%% ------------------------------------------------------------
% http://www.ituring.com.cn/article/207412 
%\usepackage{zhmakeindex}          % index processor 
%\makeindex

% <<<<<<<<<<<<<<<<<<<<<<<<<<<<<<<<<<<<<<<<<<<<<<<<<<<<<<<<<<<<<
% Page Layout
% <<<<<<<<<<<<<<<<<<<<<<<<<<<<<<<<<<<<<<<<<<<<<<<<<<<<<<<<<<<<<
% $(kpsewhich -var-value TEXMFDIST)/tex/latex/koma-script/
% http://tex.stackexchange.com/a/{19497, 19505}

%% BCOR ( binding correction / binding offset / gutter margins ) 
%% ------------------------------------------------------------ 
% 打孔裝訂 → 視打孔器型號規格而定,適合少量裝訂。裝訂膠圈容納量:
%           http://www.amazon.co.jp/dp/{B002UKPC06, B00LEAP3SO}
%           雙孔:頁邊距 13mm + 孔徑 6mm;多孔:頁邊距 3.6mm + 孔徑 5.5mm 
% 膠裝裝訂 → https://www.createspace.com/Products/Book/InteriorPDF.jsp
%
% 補充:書背厚度之計算(來自同人誌印刷廠商)
% ( 內文總頁數 ÷ 2 ) x ( 紙張條數 ÷ 100 ) + 1mm = 書背厚度(mm,無條件捨去小數點,取整數值)
% 紙張條數(單紙厚度)→ 80gsm 道林紙:10 條;100gsm 道林紙:12 條
%                    80gsm 畫刊紙: 9 條;100gsm 畫刊紙:11 條 

% $(kpsewhich -var-value TEXMFDIST)/tex/latex/koma-script/typearea.sty
% \KOMAoptions{BCOR=?, DIV=last} or 
% \typearea[BCOR]{DIV} or 
% \areaset[BCOR]{textwidth}{textheight} or 
% usegeometry
\KOMAoptions{headinclude = false, footinclude = false, mpinclude = true} 
\areaset[9mm]{16.65cm}{22cm}   % 參照別人設定,145mm x 210mm 或 160mm x 237mm 斟酌加減
%\recalctypearea               

% 字數估測:
% textwidth / font size
% 1cm = 28.45274pt

% 實測結果( XeCJK ):
% 一橫行 / 一橫列最多 )11pt | 12pt  中文字(不含標點符號)
% ( cwTeX Q Ming   )  45 | 41      
% ( Adobe Ming Std )  40 | 37
% ( AR PL UMing TW )  40 | 37
% ( BabelStone Han )  40 | 37
% ( FandolSong     )  40 | 37

% 實測結果( LuaTeX-ja ):
% 一橫行 / 一橫列最多 )11pt | 12pt  中文字(不含標點符號)
% ( cwTeX Q Ming   )   找不到       
% ( Adobe Ming Std )  42 | 38
% ( AR PL UMing TW )  42 | 38
% ( BabelStone Han )  42 | 38
% ( FandolSong     )  42 | 38 

%% Page Header(頁眉) & Footer(頁腳)
%% ------------------------------------------------------------
% http://tex.stackexchange.com/questions/126531/       
\usepackage{scrlayer-scrpage}
\automark{chapter}  % \automark deletes all prior usages of \automark or \automark*
\automark*{section} 
\clearpairofpagestyles
%\addtokomafont{pagenumber}{\sffamily \bfseries}
%\addtokomafont{pagehead}{\sffamily}
%\addtokomafont{chapter}{\sffamily \bfseries}

%雙頁且開書邊向右(偶數頁在左,奇數頁在右):

%% 頁眉(左上方):章節名
%% \lehead{\headmark}  % "le" means "left & even"

%% 頁眉(左上方):頁數 | 章節名
\lehead{%
\llap{\pagemark~~~\smash{%
  \rule[-0.4ex]{0.4pt}{\dimexpr\headheight-\topmargin+\headsep\relax}}%
  \hspace{2em}}%
\headmark}

%% 頁眉(右上方):頁數
%% \rohead{\pagemark}  % "ro" means "right & odd"

%% 頁眉(右上方):|頁數
\rohead{%
\headmark%
\rlap{\hspace{2em}\smash{%
  \rule[-0.4ex]{0.4pt}{\dimexpr\headheight-\topmargin+\headsep\relax}}%
  ~~~\pagemark}} 

%% Margin Notes
%% ------------------------------------------------------------
% $(kpsewhich -var-value TEXMFDIST)/tex/latex/koma-script/scrlayer-notecolumn.sty
% usage: \makenote[<name of notecolumn>]{<note>}
%\usepackage{scrlayer-notecolumn}

%% Chapter Tab Index
%% ------------------------------------------------------------ 
%\usepackage{chapterthumb}     % in the koma-script-examples package

%% Title Page
%% ------------------------------------------------------------ 
%\KOMAoptions{titlepage=on}
%\renewcommand\coverpagetopmargin{}
%\renewcommand\coverpagebottommargin{}
%\renewcommand\coverpageleftmargin{}
%\renewcommand\coverpagerightmargin{}

% Table of Contents ( ToC )
%% ------------------------------------------------------------ 
% $(kpsewhich -var-value TEXMFDIST)/tex/latex/koma-script/{tocbasic,tocstyle}.sty
\KOMAoptions{bibliography=totoc}
%\KOMAoptions{listofnumbered, index=totoc}
\setcounter{secnumdepth}{3}    % Level for numbered captions 
\setcounter{tocdepth}{2}       % Level of chapters that appear in ToC
% -2 no caption at all
% -1 part
% 0  chapter
% 1  section    
% 2  subsection 
% 3  subsubsection
% 4  paragraph
% 5  subparagraph

%% Headings Style
%% ------------------------------------------------------------ 
% http://tex.stackexchange.com/a/240569
% \RedeclareSectionCommand[attributes]{name} 

%%% T i t l e
%%% ===========================================================
%\setkomafont{title}{}
%\setkomafont{subtitle}{}
%\setkomafont{author}{}
%\setkomafont{date}{}

%%% P a r t
%%% ===========================================================
\setkomafont{part}{\normalfont\bfseries\huge}
\setkomafont{partnumber}{\normalfont\bfseries\Huge}
\RedeclareSectionCommand[innerskip=10pt]{part}
\renewcommand\partformat{%
  \strut\MakeUppercase{\lsstyle% Upper case sequences should be spaced
    \partname~\thepart}%
  \vspace{5pt}\hrule height 3pt%
}

%%% C h a p t e r
%%% ===========================================================
\KOMAoptions{chapterprefix}
\renewcommand\raggedchapter{\centering}
\setkomafont{chapter}{\normalfont\FANGSONG\LARGE}
%\setkomafont{chapter}{\normalfont\bfseries\LARGE}
%\setkomafont{chapterprefix}{\huge}
\setkomafont{chapterprefix}{\bfseries\huge}
\RedeclareSectionCommand[beforeskip=0pt,afterskip=40pt,innerskip=12pt]{chapter}
\renewcommand\chapterformat{%
  \hrule height 3pt\vspace{3pt}\hrule height 1pt\vspace{5pt}%
  \mbox{\strut\MakeUppercase{\lsstyle% Upper case sequences should be spaced
      \chapapp\nobreakspace\thechapter}}%
  \hrule height 1pt%
}

%%% S e c t i o n
%%% ===========================================================
%\setkomafont{section}{}
%\setkomafont{subsection}{}

%%% T e x t    B o x e s
%%% ===========================================================

% <<<<<<<<<<<<<<<<<<<<<<<<<<<<<<<<<<<<<<<<<<<<<<<<<<<<<<<<<<<<<
% Page Layout( END )
% <<<<<<<<<<<<<<<<<<<<<<<<<<<<<<<<<<<<<<<<<<<<<<<<<<<<<<<<<<<<<
       
%% PDF metadata & Hyperlinks 
%% ------------------------------------------------------------ 

% m e t a d a t a
\title{文件標題}
\author{林子楀( Lin Tzu-Yu )}
\date{\today}                                % or \zhtoday ( zhnumber package )

\def\docTitle{文件標題測試文件標題測試}         % for custom titlepage
\def\docAuthor{林子楀( Lin Tzu-Yu )}        
\def\docDate{\zhtoday}

% h y p e r r e f
% ftp://tug.ctan.org/tex-archive/macros/latex/contrib/hyperref/doc/options.pdf 
% 顏色( texdoc xcolor ):
% - 黑 Black
% - 橘 DarkOrange 
% - 紅 Crimson, DeepPink, VioletRed
% - 紫 DarkOrchid,
% - 藍 DodgerBlue, RoyalBlue, SteelBlue
% - 綠 SeaGreen, Teal 
\ifXeTeX
  \usepackage[setpagesize=false,             % page size defined by xetex
              unicode=false,                 % unicode breaks when used with xetex
              xetex]{hyperref}
\else
  \usepackage[unicode=true]{hyperref}
\fi
\hypersetup{          
    pdfpagemode        = UseOutlines,
    pdffitwindow       = true,                % window fit to page when opened
    pdfstartview       = {FitH},              % fits the width of the page to the window
    pdfborder          = {XYZ null null 1},
    breaklinks         = true,
    bookmarksopen      = true,
    bookmarksnumbered  = true,
    bookmarksopenlevel = 0,
    colorlinks         = true,                % false: boxed links
    allcolors          = Crimson}
    
\makeatletter
  \hypersetup{
    pdftitle           = {\@title},
    pdfauthor          = {\textcopyright \@author}
  }
\makeatother

% u r l
%\usepackage[hyphens]{url}
\usepackage{url}

% Break URLs in a few more places. (For breaking lines, not functionality!)
% cf. http://tex.stackexchange.com/questions/3033/forcing-linebreaks-in-url
\expandafter\def\expandafter\UrlBreaks\expandafter{\UrlBreaks%  save the current one
  \do\*\do\-\do\~\do\'\do\"\do\-}

%% Testing
%% ------------------------------------------------------------
%\usepackage{layouts}
%\usepackage{showframe}             % shows typing area and margins
\usepackage{blindtext}              % dummy text
\usepackage{mwe}                    % dummy graphics

% ****************************************************************************
% Content
% ****************************************************************************
\begin{document}

% <<<<<<<<<<<<<<<<<<<<<<<<<<<<<<<<<<<<<<<<<<<<<<<<<<<<<<<<<<<<<
% FRONT MATTER
% <<<<<<<<<<<<<<<<<<<<<<<<<<<<<<<<<<<<<<<<<<<<<<<<<<<<<<<<<<<<<
%\frontmatter

%% Book Cover, Title Page and Copyright Page
%% ------------------------------------------------------------
%\pagenumbering{roman}              % roman: lower-case Roman numbers ex. viii
%\pagestyle{empty}                  % no header or footers
% !TEX root = ../main.tex
% cf. Clean Thesis http://cleanthesis.der-ric.de/

%% Cover Page
%% ------------------------------------------------------------

%% Title Page
%% ------------------------------------------------------------
\begin{titlepage}
	\pdfbookmark[0]{Cover}{Cover}
	\flushright
	\hfill
	\vfill
	{\FANGSONG\LARGE\docTitle \par}
	\rule[5pt]{\textwidth}{.4pt} \par
	{\FANGSONG\Large\docAuthor}
	\vfill
	\textrm{\large\docDate} \\ 
\end{titlepage}

%% Copyright Page
%% ------------------------------------------------------------
\newpage
~\vfill
\thispagestyle{empty}
\noindent Copyright \copyright\ 2016 \docAuthor \\
\noindent Licensed under the Creative Commons Attribution-NonCommercial-ShareAlike 4.0 International License (the ``License'').
You may not use this file except in compliance with the License. You may obtain a copy of the License at \url{https://creativecommons.org/licenses/by-nc-sa/4.0/}. \\ 
 

%% Dedication, Abstract, Preface, Acknowledgments, ...
%% ------------------------------------------------------------
%\pagestyle{plain}                  % display just page numbers  
%\input{}

%% Table of Contents ( ToC )
%% ------------------------------------------------------------
\renewcommand{\contentsname}{目錄}  
\tableofcontents
%\listoffigures                      % List of Figures ( lof )
%\listoftables                       % List of Tables ( lot )

% <<<<<<<<<<<<<<<<<<<<<<<<<<<<<<<<<<<<<<<<<<<<<<<<<<<<<<<<<<<<<
% BODY MATTER ( Content )
% <<<<<<<<<<<<<<<<<<<<<<<<<<<<<<<<<<<<<<<<<<<<<<<<<<<<<<<<<<<<<
%\mainmatter
%\pagenumbering{arabic}             
%\input{content.tex} 
\part{緒論}
%% Lorem Ipsum
%% ------------------------------------------------------------
\chapter{Chapter 1}
% Chapter quote at the start of chapter   
%\chapterquote{This is a quote text.}{Author’s name}{(Source of this quote)}
\blindtext[2]
\section{Section 1}
\blindtext[1]
\subsection{Subsection 1}
\blindtext[1]
\subsubsection{Subsubsection 1}
\blindtext[1]
\chapter{Chapter 2}
\blindtext[3]

%% 特殊符號
%% ------------------------------------------------------------
\chapter{Symbols}

% &
\&

% §
\S

% #
\#

% %
\%

% …
\ldots

% |
\textbar

% ●
\textbullet 

% \
\textbackslash

% →
\rightarrow

\Rightarrow

\faArrowRight     % fontawesome

% ←→
\leftrightarrow

\Leftrightarrow

\faArrowsH        % fontawesome

% _
\_

\section{Dashes}

% hyphen ( in words )
X-ray

% en-dash ( between numbers )
1--10

% em-dash ( punctuation )
Yes---or no?


%% 中文
%% ------------------------------------------------------------
\chapter{測試長標題測試長標題測試長標題測試長標題測試長標題}
\section{數字}
一二三四五六七八九十壹貳參肆伍陸柒捌玖拾一二三四五六七八九十壹貳參肆伍陸柒捌玖拾一二三四五六七八九十

\zhdigits[styl=Financial]{1234567890}

\zhnum{1234567890}

\zhnumber{1234567890}

%\section{字體}

%\HEI{黑體}\KAI{楷書}\FANGSONG{仿宋}

\section{沒有中文假文產生器好麻煩}
街上頓添一種活氣,也就不容易,一年的設定,這不是和舊時代的天子,是怨是讎?他倆疲倦了,草繩上插的香條,但這是所謂大勢,不知流失多少人類所託命的田,甲吐出那飽吸過的香煙,由深藍色的山頭,好,街上看鬧熱的人,也因為能得到較多的金錢,波湧似的,不知行有多少時刻,福戶內的事,我記得還似昨天,自己走出家來,街上佈滿著倦態和睡容,所能喚起的兒童時代,來--來!音響的餘波,把她清冷冷的光輝,一絲絲涼爽秋風,僅七、八歲時的狀況,橋柱是否有傾斜,誰甘白受人家的欺負,樹要樹皮人要麵皮,鑼聲亦不響了,不知是兄哥或小弟,他就發出歡喜的呼喊,也是不受後母教訓,可能成功嗎?只些婦女們,是社會的一成員,這不是和舊時代的天子,使成粉末,很好,但終於覺悟地走向滅亡,放不少鞭炮,光明已在前頭,也看得見濃墨一樣高低的樹林,街上還是鬧熱,那些富家人,在我回憶裡,悠揚地幾聲洞簫,看他有多大力量能夠反對!這指定一箇日子為過年,早幾點鐘解決,在以前任怎地追憶,猶在戀著夢之國的快樂,蔥惶回顧,將要失去明視的效力,這是什麼言語?暗黑的氣氛,對男人怕失了他的玩愛,連生意本,甲微喟的說,因為久慣於暗黑的眼睛,何以故過年就要如此呢?\footnote{\url{http://more.handlino.com/?corpus=laihe} {MoreText.js: 一用就愛上的中文假文產生器}}

\section{quotation}
\begin{quotation}
\noindent quotation環境測試,quotation 環境測試。quotation環境測試,quotation 環境測試。quotation環境
測試,quotation 環境測試。quotation環境測試,quotation 環境測試。quotation環境測試,quotation 環境測試。

quotation環境測試,quotation 環境測試。quotation環境測試,quotation 環境測試。quotation環境
測試,quotation 環境測試。quotation環境測試,quotation 環境測試。quotation環境測試,quotation 環境測試。
\end{quotation}

\subsubsection{換行}
街上頓添一種活氣,也就不容易,\\一年的設定,這不是和舊時代的天子,是怨是讎?\\他倆疲倦了,草繩上插的香條,\\但這是所謂大勢,不知流失多少人類所託命的田,\\甲吐出那飽吸過的香煙,由深藍色的山頭,\\好,街上看鬧熱的人,也因為能得到較多的金錢,波湧似的,\\不知行有多少時刻,福戶內的事,我記得還似昨天,自己走出家來

街上頓添一種活氣,也就不容易,\\*一年的設定,這不是和舊時代的天子,是怨是讎?\\*他倆疲倦了,草繩上插的香條,\\*但這是所謂大勢,不知流失多少人類所託命的田,\\*甲吐出那飽吸過的香煙,由深藍色的山頭,\\*好,街上看鬧熱的人,也因為能得到較多的金錢,波湧似的,\\*不知行有多少時刻,福戶內的事,我記得還似昨天,自己走出家來

\ifXeTeX 
\section*{XeCJK}  % *(asterisk,米字鍵) 表示不計數
\subsection*{測試 CJKunderdot}
街上頓添一種活氣,也就\emph{不容易},\CJKunderdot{一年的設定,這不是和舊時代的天子,是怨是讎?}
\fi

\ifLuaTeX
\section*{LuaTeX-ja}
\subsection*{Ruby}

測試\ruby{紐約}{New York}\ruby{市}{City}。

「
\ruby{柴}{ㄔㄞˊ}
\ruby{米}{ㄇㄧ˅}
\ruby{油}{ㄧㄡˊ}
\ruby{鹽}{ㄧㄢˊ}
\ruby{醬}{ㄐㄧㄤˋ}
\ruby{醋}{ㄘㄨˋ}
\ruby{茶}{ㄔㄚˊ}
」這些都是我們日常生活必備的七樣東西。
\fi


\part{第二部}
%% List
%% ------------------------------------------------------------
\chapter{List}

\section{Ordered List}
1. Lorem ipsum dolor sit amet, consectetur\footnote{ \url{www.example.com}} adipiscing elit.

2. Duis ac mi magna, a consectetur elit\footnote{ Plain text.}.

3. Curabitur posuere erat \emph{dignissim ligula euismod} ut euismod nisi.

%% ------------------------------------------------------------
\subsection{enumerate}
\begin{enumerate}
  \item 第一項
  \item 第二項
  \item 第三項
  \begin{enumerate}
    \item 第一層嵌套
    \begin{enumerate}
      \item 第二層嵌套
      \begin{enumerate}
        \item 第三層嵌套(最多嵌套三層)
      \end{enumerate}
    \end{enumerate}
  \end{enumerate}
  \item 第四項
\end{enumerate}
%% ------------------------------------------------------------
\subsubsection{自訂}
\begin{enumerate}
\item[一、] 來–來!音響的餘波,把她清冷冷的光輝,一絲絲涼爽秋風,僅七、八歲時的狀
況,橋柱是否有傾斜,誰甘白受人家的欺負,樹要樹皮人要麵皮,鑼聲亦不響了,不知是兄哥或
小弟,他就發出歡喜的呼喊,放不少鞭炮,光明已在前頭,也看得見濃墨一樣高低的樹林,街上還是鬧熱。

\item[二、] 測試
  \begin{enumerate}
    \item[(一)] 第一子題
    \item[(二)] 第二子題
  \end{enumerate}

\item[三、] 第三題。
\end{enumerate}
%% ------------------------------------------------------------
\section{Unordered List}
\begin{itemize}
  \item 不嵌套
  \begin{itemize}
    \item 第一層嵌套
    \begin{itemize}
      \item 第二層嵌套
      \begin{itemize}
        \item 第三層嵌套(最多嵌套三層)
      \end{itemize}
    \end{itemize}
  \end{itemize}
  \item 不嵌套
\end{itemize}

\begin{itemize}
  \item \blindtext[1]
  \item \blindtext[1]
  \item \blindtext[1]
  \begin{itemize}
    \item 第一層嵌套
    \begin{itemize}
      \item 第二層嵌套
      \begin{itemize}
        \item 第三層嵌套(最多嵌套三層)
      \end{itemize}
    \end{itemize}
  \end{itemize}
  \item \blindtext[1]
\end{itemize}
%% ------------------------------------------------------------
\section{Description List}
\begin{description}
    \item[Biology] Study of life.
    \item[Physics] Science of matter and its motion.
    \item[Psychology] Scientific study of mental processes and behaviour.
\end{description}

%% ------------------------------------------------------------
\section{Nested lists}
\begin{itemize}
    \item First level, itemize, first item
    \begin{itemize}
        \item Second level, itemize, first item
        \item Second level, itemize, second item
        \begin{enumerate}
            \item Third level, enumerate, first item
            \item Third level, enumerate, second item
        \end{enumerate}
    \end{itemize}
    \item First level, itemize, second item
\end{itemize}


%% Listings
%% ------------------------------------------------------------
\chapter{Listings}
\section{一般}
\begin{lstlisting}[language=Python, caption=This is caption of code block]

 ''' A multi-line
     comment.'''
 def sub_word(mo):
     ''' Single line comment.'''
     word = mo.group('word')   # Inline comment
     if word in keywords[language]:              
         return quote + word + quote              
     else:                                       
         return word
\end{lstlisting}

\section{嵌入}
\lstinputlisting[label=src:hello,language=Ruby,caption=The Greeter class]{files/hello.rb}

%% Graph
%% ------------------------------------------------------------
%\chapter{圖}
%\section{Figure}
%% ------------------------------------------------------------
\begin{figure}[h!]
\centering
\includegraphics[
  height=\dimexpr\pagegoal-\pagetotal-4\baselineskip\relax,
  width=0.6\textwidth,
  keepaspectratio]
{example-image}
\caption{figure name}
\label{fig:key}
\end{figure}

\subsection{Subfigure}
%% ------------------------------------------------------------
\begin{figure}
    \includegraphics[width=.48\linewidth]{example-image-a}\hfill
    \includegraphics[width=.48\linewidth]{example-image-b}
    \caption{MWE to demonstrate how to place to images side-by-side}
\end{figure}
%\section{fast-diagram}
\begin{fast}{動物}
\FT{哺乳類}{\FT{人}{} \FT{鯨}{}}
\FT{兩生類}{\FT{山椒魚}{} \FT{鮭}{}}
\FT{魚 類}{\FT{鮫}{} \FT{秋刀魚}{}}
\end{fast}

source: \url{http://konoyonohana.blog.fc2.com/blog-entry-131.html}

%% 對我來說太複雜。

\section{pgfplots}
%% ------------------------------------------------------------
\begin{tikzpicture}
   \begin{loglogaxis}[
        title =Convergence Plot,
        xlabel={Degrees of freedom},
        ylabel={$L 2$ Error},
        grid=major,
        legend entries ={$d=2$,$d=3$,$d=4$},
    ]
        \addplot table {data d2.dat};
        \addplot table {data d3.dat};
        \addplot table {data d4.dat};
        \addplot table[
            x=dof,
            y={create col/linear regression ={y=l2 err,
            variance list ={1000,800,600,500,400,200,100}}}
        ] {data d4.dat};
    \end{loglogaxis}
\end{tikzpicture}

%% ------------------------------------------------------------
\begin{tikzpicture}
    \begin{axis}[
        title =Inv. cum. normal,
        xlabel={$x$},
        ylabel={$y$},
        ymin=−3, ymax=3,
        minor y tick num=1,
    ]
        \addplot[blue] table {./invcum.dat};
    \end{axis}
\end{tikzpicture}

%% ------------------------------------------------------------
\begin{tikzpicture}
    \begin{axis}[
        domain=0:10,scaled ticks=false,
        ymax=2000,ymin=2,minor tick num=1,
        xlabel=$x$, ylabel=$f(x)$,
    ]
        \addplot+[no marks,thick] {xˆ3};
        \addlegendentry{$xˆ3$}; 
        \addplot+[no marks,thick] {x∗(2)ˆx};
        \addlegendentry{$2ˆx\,x$};
    \end{axis}
\end{tikzpicture}

%% ------------------------------------------------------------

\begin{tikzpicture}
    \pgfmathdeclarefunction{sincf}{1}{%
        \pgfmathparse{(abs(#1)<0.01) ? 1 : sin(pi∗#1 r)/(pi∗#1)}%
    }
    \begin{axis}
        \addplot {sincf(\x)};
    \end{axis}
\end{tikzpicture}

%% ------------------------------------------------------------
\begin{tikzpicture}
    \begin{axis}[width=8cm,
        axis lines =center,
        xmin=3,xmax=19,
        ymin=0,ymax=3,
        grid=major,
        yminorgrids,
        ylabel near ticks ,
        xlabel near ticks ,
        major grid style ={thick},
        tick style ={thick},
        axis line style ={thick},
        xlabel=Age,
        ylabel=Frequency density,
        xtickmin=4,
        xtick ={5,6,...,18},
        ytick={0,1,2,3},
        minor y tick num=3,
        axis x discontinuity =crunch,
        ticklabel style ={font=\tiny},
        enlarge x limits ={upper,value=0.02},
        enlarge y limits ={upper,value=0.1}
    ]
    \addplot[ybar interval , fill =red!80!black,draw=white]
        coordinates {(5,1)(11,3)};
    \addplot[ybar interval , fill =orange!80!yellow,draw=white]
        coordinates {(11,3)(16,2)};
    \addplot[ybar interval , fill =cyan,draw=white]
        coordinates {(16,2)(18,2)};
    \end{axis}
\end{tikzpicture}

source: \url{http://konoyonohana.blog.fc2.com/blog-entry-140.html}

%% Tables
%% ------------------------------------------------------------
%\chapter{表}
%% ------------------------------------------------------------
\begin{table}[htbp]
    \centering                  % used for centering table
    \begin{tabular}{c c c c}    % centered columns (4 columns)
        \hline\hline                
        Col\#1 & Col\#2 & Col\#3 & Col\#4 \\ [0.5ex]
        \hline                      
        1 & 一 & A & 壹 \\
        2 & 二 & B & 貳 \\
        3 & 三 & C & 參 \\ [1ex]    % [1ex] adds vertical space
        \hline                      
    \end{tabular}
    \caption{Sample table}         % title of Table
    \label{tab:123table}           % is used to refer this table in the text
\end{table}

%% ------------------------------------------------------------
\section{三線表}
\begin{table}
  \centering
  \renewcommand{\arraystretch}{1.6}
  \begin{tabular}{lcc}
    \toprule
        col1 & col2 & col3 \ \
    \cmidrule(r){1-1}\cmidrule(lr){2-2}\cmidrule(l){3-3}          
        A & 1 & One   \ \
        B & 2 & Two   \ \
        C & 3 & Third \ \
    \bottomrule
  \end{tabular}
  \caption{三線表}
  \label{tab:三線表}
\end{table}

%% ------------------------------------------------------------
%\section{Vertical TimeLine}
\begin{table}
\centering                                                         % 居中
\renewcommand\arraystretch{1.4}\arrayrulecolor{base0D}
\caption*{Timeline}                           
\begin{tabular}{@{\,}r <{\hskip 2pt} !{\vline} >{\raggedright\arraybackslash}p{6cm}}
                                                                   % 表格總寬 6cm 
\toprule
\addlinespace[1.5ex]                                               % 表格行距
1947 & AT and T Bell Labs develop the idea of cellular phones\\    % 年份 & 事件 \\(換行) 
1968 & Xerox Palo Alto Research Centre envisage the 'Dynabook\\
1971 & Busicom 'Handy-LE' Calculator\\
1973 & First mobile handset invented by Martin Cooper\\
1978 & Parker Bros. Merlin Computer Toy\\
1981 & Osborne 1 Portable Computer\\
1982 & Grid Compass 1100 Clamshell Laptop\\
1983 & TRS-80 Model 100 Portable PC\\
1984 & Psion Organiser Handheld Computer\\
1991 & Psion Series 3 Minicomputer\\
\end{tabular}


%% Cross References
%% ------------------------------------------------------------
\chapter{Cross References}
\section{Footnotes}
1. Lorem ipsum dolor sit amet, consectetur\footnote{ \url{www.example.com}} adipiscing elit.

2. Duis ac mi magna, a consectetur elit\footnote{ Plain text.}.

3. Curabitur posuere erat \emph{dignissim ligula euismod} ut euismod nisi.


\section{測試引註}

測試supercite\supercite{Mackay:2012}

測試parencite\parencite{行政契約2013}

測試cite\cite{biber},又如\cite[文獻][第 10 頁]{biber}

同時引用\cite{usdsbook, obama}

\nocite{*}

\section{引用}
% 先設 \label{key},為了避免重複,
% Chapter 可用 \label{cha:key};Section 可用 \label{sec:key};例:\chapter{name}\label{key}
% Table 可用 \label{tab:key};Figure 可用 \label{fig:key};Source Code 可用 \label{src:key}
引用 "The Greeter class" 程式碼 \ref{src:hello},在第 \pageref{src:hello} 頁。

%引用三線表 \ref{tab:三線表},在第 \pageref{tab:三線表} 頁。


% <<<<<<<<<<<<<<<<<<<<<<<<<<<<<<<<<<<<<<<<<<<<<<<<<<<<<<<<<<<<<
% BACK MATTER ( Appendix )
% <<<<<<<<<<<<<<<<<<<<<<<<<<<<<<<<<<<<<<<<<<<<<<<<<<<<<<<<<<<<<
%\backmatter
\appendix

%% Glossaries
%% ------------------------------------------------------------

%% Index
%% ------------------------------------------------------------
%\printindex

%% Bibliography & References
%% ------------------------------------------------------------
\printbibliography[headings=參考文獻]
\end{document}
